\chapter{Introduction}
\label{chap:intro}

By shedding light on the transformative potential of canisters and blockchain technology, this research aims to inspire innovation, facilitate secure and transparent digital transactions, and contribute to the advancement of blockchain and web application development as a whole.

This thesis seeks to address these challenges by exploring the integration of blockchain technology, specifically through the use of canisters, into web application development. Canisters, as self-contained units of computation on the Internet Computer, offer a unique framework for building decentralized applications. By leveraging the power of canisters, developers can create web applications that benefit from the security, transparency, and decentralized nature of blockchain technology.

\section{Aim of the Project} \label{sec:s1}
With the wide popularity of Internet and related technologies, various Industry 4.0-based applications have been used across the globe in which sensors and actuators sense, compute and communicate the data for industry automation. As in Industry 4.0-based applications, data between different locations flows using an open channel, i.e., Internet, so threats to security and privacy has also increased manyfold. Such applications deal with data in large volumes and hence, so it is necessary to consider issues such as-data heterogeneity, data integrity, and data redundancy alongwith the security and privacy concerns. Moreover, different applications require datasets from different domains in different formats. Therefore, it is also needed to standardize the data format so that it can be used by different Industry 4.0-based applications.
% \ac{ac}, some citation \cite{citeKey1}, and some more \ac{ac2}.

\section{Motivation}
In today's digital landscape, the potential of blockchain technology to transform various industries has captured significant attention. Blockchain's decentralized and immutable nature offers new possibilities for secure and transparent transactions, data management, and trust-building mechanisms. However, despite its promises, the adoption of blockchain in practical applications still faces challenges, particularly in the realm of web application development.

Traditional web applications often rely on centralized architectures, which can be vulnerable to single points of failure, data breaches, and lack of transparency. Additionally, scalability and performance limitations have hindered the widespread adoption of blockchain in web development. These challenges have sparked the need for innovative approaches that combine the benefits of blockchain technology with efficient web application design.

The motivation for this research lies in the potential of canisters to revolutionize web application development. By incorporating blockchain principles and utilizing canisters as autonomous computational units, developers can overcome the limitations of centralized architectures. Canisters provide a scalable, secure, and tamper-resistant environment that enables the implementation of innovative web applications with enhanced privacy, data integrity, and user control.

Furthermore, this thesis aims to contribute to the existing body of knowledge by investigating the practical implications, challenges, and opportunities associated with utilizing canisters in blockchain-powered web applications. By examining real-world use cases and evaluating the performance and scalability of canister-based web applications, this research seeks to provide valuable insights for developers, businesses, and researchers interested in harnessing the potential of blockchain technology.

Ultimately, the findings of this research endeavor can pave the way for the widespread adoption of blockchain-powered web applications, transforming industries such as finance, supply chain, healthcare, and beyond. By bridging the gap between blockchain technology and web application development, this thesis aims to empower developers and organizations to embrace the decentralized paradigm and unlock the full potential of blockchain in the context of web applications.

\section{Thesis Organization}
To address the aim of our project and delve into the motivations behind it, this thesis is organized as follows:

\subsubsection{Chapter 2: Background}
In this chapter, we lay the foundation by providing a comprehensive overview of the background information related to our research. We explore the architecture of blockchain technology, including important concepts like the Merkle hash tree and Elliptic Curve Cryptography. We also dive into the double spending problem, showcasing its real-world illustration, examples, and existing solutions. Additionally, we discuss the diverse applications of blockchain, such as its impact on transportation, climate change, and healthcare. Furthermore, we explore smart contracts, explaining how they work and highlighting prominent platforms like Ethereum and Canister. Finally, we touch upon decentralized applications (DApps) and their significance in the blockchain ecosystem.

\subsubsection{Chapter 3: Methodology}
This chapter outlines the approach we followed to achieve the objectives of our project. We begin by describing our careful selection process of the platform we worked with. We then provide insights into the system design phase, where we conceptualized and planned the technical aspects of our solution. Additionally, we detail the implementation process, discussing the practical steps we took to bring our solution to life. This chapter offers a comprehensive view of the methodologies employed throughout our research.

\subsubsection{Chapter 4: Experimental Results}
Here, we present the results obtained from the experiments we conducted as part of our research. Our focus is on evaluating the transaction speed, a crucial metric that helps us gauge the efficiency of our proposed solution. By analyzing the experimental data, we gain insights into the performance and effectiveness of our system.

\subsubsection{Chapter 5: Conclusion}
In this final chapter, we summarize the key findings and contributions of our research. We provide a concise overview of the entire thesis, highlighting the implications of our results. Additionally, we identify areas for future work and improvements, specifically addressing the need to enhance the update call performance and simplify the deployment of web applications on Canisters and the Internet Computer. This chapter concludes our thesis, emphasizing the significance of our research and its potential for future advancements.