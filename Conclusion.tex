\chapter{Conclusion}
\label{chap:todo}
In conclusion, this project has made significant contributions to the field of blockchain and decentralized applications. By improving transaction speed and developing a secure smart contract execution framework, we have addressed key challenges in blockchain technology. However, there is still much to explore in terms of scalability, privacy, and interoperability. The future works outlined above provide valuable directions for researchers and developers to further advance the capabilities and applicability of blockchain technology.
\section{Main Contributions}
In this thesis, we have accomplished some significant achievements in the field of web application development using canisters and blockchain technology. Let me break down our contributions for you:

\begin{enumerate}
    \item  Delving into the world of blockchain architecture: We embarked on a comprehensive exploration of the architecture of blockchain, uncovering its decentralized nature and the fundamental principles that govern its operation. This understanding forms the bedrock for utilizing canisters as a decentralized approach to web application development.

    \item  Solving the double spending problem: We tackled a critical issue in blockchain networks—double spending—and delved into innovative solutions to mitigate this problem. Through our analysis and evaluation of these solutions, we have added to the existing knowledge about the security and integrity of blockchain-based systems.

    \item  Discovering the diverse applications of blockchain: We went on a journey to explore the diverse applications of blockchain technology across various industries, including climate, supply chain management, and healthcare. Our investigation shed light on how blockchain can bring transformative changes to these sectors, providing valuable insights for researchers, businesses, and decision-makers.

    \item  In-depth research on canisters: We dedicated extensive effort to conducting in-depth research and practical analysis of canisters. We focused on their architecture, programming model, and deployment procedures to gain a comprehensive understanding of how to utilize them in web application development. Our findings contribute to the technical knowledge required for effectively leveraging canisters.

    \item  A captivating case study: To demonstrate the real-world potential of canisters, we developed a fully functional web application. This case study not only showcased the advantages of using canisters but also highlighted the complexities involved. We tackled challenges such as data storage, user authentication, and transaction management, providing valuable insights for developers and businesses.
\end{enumerate}

Overall, our research expands the existing knowledge in decentralized computing and emphasizes the transformative power of canisters in web application development. Our insights and findings aim to provide practical guidance and inspiration for developers, businesses, and end-users who are eager to harness the benefits of blockchain technology in their innovative web-based solutions.
\section{Future Work}
While this project has achieved significant progress in the field of blockchain and decentralized applications, there are several areas that warrant further exploration and improvement. The following future works outline potential directions to enhance the update call performance and facilitate the deployment of web applications on canisters and the Internet Computer:

\begin{enumerate}
    
    \item Optimization of Update Call Speed
    
One crucial area for future research is the optimization of update call speed in canisters. Despite the advancements made in smart contract execution, there is still room for improving the efficiency of update calls. Exploring novel techniques, such as parallel processing, caching mechanisms, or optimizing the communication protocols between canisters and clients, can help reduce the latency and improve the responsiveness of update calls. By enhancing the update call speed, we can enable faster and more interactive decentralized applications.


    \item Streamlining Web Application Deployment on Canisters
    
Another important direction for future development is streamlining the deployment process of web applications on canisters and the Internet Computer. Simplifying the steps required to deploy web applications, including packaging dependencies, configuring network access, and managing application state, can lower the barrier to entry for developers and encourage broader adoption of canister-based applications. Investigating automated deployment tools, integrating with popular web frameworks, and providing comprehensive documentation can facilitate the seamless deployment of web applications on the Internet Computer.
\end{enumerate}